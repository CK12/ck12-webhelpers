%%% ck12_common LaTeX commands reside here.  Th 
%%% FILE: ck12_common.tex
%%% DESC: Common macros and functions used for the one/two-column templates and
%%%       with f2pdf.py.  To be included at in the individual template files' header.
%%% AUTH: Britton Olson
%%% DATE: Aug. 30, 2011

%%%%%%%%%%%%%%%%%%%%%%%%%%%%%%%%%%%
%%%% Package declarations here %%%%
%%%%%%%%%%%%%%%%%%%%%%%%%%%%%%%%%%%
\usepackage{amssymb}
\usepackage{amsmath}
\usepackage{txfonts}   %% needed for black heart and diamond suit symbols

\usepackage{textcomp}
\usepackage{ifthen}
\usepackage{calc}
\usepackage{fullpage}
\usepackage{enumitem}
\usepackage{fancyhdr}
\usepackage{breqn}     % Break equation package.... nice

\usepackage{graphicx}
\usepackage{multicol}  % Used for twocolumn template
\usepackage{float}
\usepackage{morefloats}
\usepackage{afterpage}

\usepackage{ltxtable}
\usepackage{caption}
\captionsetup[figure]{labelformat=empty, labelsep=none} % This will supress the default caption for figures
\usepackage{filecontents}
\usepackage{booktabs}
\usepackage{lscape}

\usepackage{minitoc}
\usepackage{tocloft}
\setlength{\cftsecnumwidth}{40 pt}
\cftpagenumbersoff{sec}


\usepackage[vertfit]{breakurl}
\usepackage{grffile}

% Definition for filled suits (diamonds and hearts)
\DeclareSymbolFont{extraup}{U}{zavm}{m}{n}
\DeclareMathSymbol{\varheart}{\mathalpha}{extraup}{86}
\DeclareMathSymbol{\vardiamond}{\mathalpha}{extraup}{87}

% For the ``permille'' symbol
\usepackage{wasysym}

\usepackage{hyperref}
\hypersetup{
    colorlinks=true,        % false: boxed links; true: colored links
    linkcolor=blue,        % color of internal links
    urlcolor=blue,           % color of external links
    breaklinks=true
}
\usepackage[all]{hypcap}    % Used to fix ref. click to image function

\usepackage[utf8x]{inputenc}  % Encoding for utf8 characters
\usepackage[encapsulated]{CJK}
\PrerenderUnicode{}
%\usepackage{xunicode}

%% JG's request (i think color is already included from hyperref)
\usepackage{color}
\usepackage{cancel}
\usepackage{setspace}
\usepackage{titling}
    
\setlength{\parindent}{0em}
\setlength{\parskip}{0.5em}
\setcounter{secnumdepth}{2}
\setcounter{chapter}{0}

\setlist{noitemsep}

\usepackage{placeins}


%%%%%%%%%%%%%%%%%%%%%%%%%%%%%%%%%%%%%%%%%%%%%%%%%%
%%%%  Command definitions and functions here  %%%%
%%%%%%%%%%%%%%%%%%%%%%%%%%%%%%%%%%%%%%%%%%%%%%%%%%


\def\UrlFont{\normalsize}  % Perhaps use URL for links as hyperef spills over columns

% Macro for ck12 colored logo
\newcommand{\cklogo}{{\color{black}www.}{\color{ckorange}c}{\color{ckgreen}k12}{\color{black}.org}}


%% For figure captions 
\captionsetup{format=plain,font=large}
\captionsetup{labelfont={Large,sc,bf,color=ckgreen,sf}} % Add "sc" for all caps
\captionsetup{textfont={sf}}



%% For the customized section titles
\usepackage[explicit]{titlesec}
\usepackage[T1]{fontenc}
\renewcommand{\sectionmark}[1]{\markright{\thesection.\ #1}}

\newcommand*\chapterlabel{}         % This is 'title'
\newcommand*\sectionlabel{}         % This is 'h1'
\newcommand*\subsectionlabel{}      % This is 'h2'
\newcommand*\subsubsectionlabel{}   % This is 'h3'


\usepackage{tikz}
\usetikzlibrary{arrows,shapes,shadows,patterns,fadings}


\usepackage{pslatex}
\usepackage{contour}
\contourlength{1pt} %how thick each copy is
\contournumber{20}  %number of copies
\newcommand{\outL}[2]{\contour{black}{\textcolor{#1} {#2}}}


\titleformat{\chapter}
  { \color{white}\tiny } {} {} {} {}

%% Long Fancy Section: For Long titles that may spill over, use this function
%% \Custom section
%%%\newcommand{\CKchapter}[2]{
%%%  \chapter{#1}
%%%  \vspace*{-2in}
%%%  % and now format the header according to spec.
%%%  \begin{tikzpicture}[remember picture,overlay]
%%%    \filldraw[top color=white,bottom color=ckgreen, draw=white] (-60pt, -.5in) rectangle (\paperwidth ,.5in); %3cm);
%%%    \node[anchor=west,yshift=0pt,xshift=-30pt] {\sffamily\Huge\bfseries\scshape\fontsize{60}{72}\selectfont \outL{ckorange}{\thechapter}  };
%%%    \node[anchor=east,xshift=.9\paperwidth,yshift=-.5in,rectangle,
%%%      %rounded corners=20pt,inner sep=11pt,fill=ckorange,minimum height=36pt]
%%%      rounded corners=20pt,inner sep=11pt,left color=ckorange,right color=ckorange, minimum height=36pt]
%%%         {#2\normalfont\sffamily\bfseries\scshape\color{white} #1\fontsize{30}{36}\selectfont\color{ckorange}{ }};
%%%  \end{tikzpicture}
%%%  \vspace*{.65in}
%%%}

\newcommand{\CKchapter}[2]{
  \chapter{#1}
  \vspace*{-144pt}
  \begin{tikzpicture}[remember picture,overlay]
  \node at (current page.north west) [anchor=west,xshift=0.75in, yshift=-48pt] (dummy) {};
    \draw [ color=ckgreen, line width=2pt] (dummy.west) --++ ( 7.0in , 0pt);   
    \node at (dummy.west) [anchor=north west, xshift=8pt , yshift=0pt] (chap) { \scshape\fontsize{20}{24}\selectfont \chaptername };
    \node at (chap.north east) [anchor=north west, xshift=2pt , yshift=0pt,rectangle,top color=ckgreen,bottom color=ckgreen!80!white ] {\sffamily\Huge\bfseries\scshape\fontsize{50}{60}\selectfont\color{black} {\thechapter}  };
  \end{tikzpicture}
  \vspace*{16pt}  
  \begin{flushright}  { #2 \normalfont\sffamily\bfseries\color{black} ~ ~ ~ ~ ~ ~ ~ ~ ~ \NOhyph{#1}} \end{flushright}
  \vspace*{0pt}
}

% DEPRECATED
\newcommand{\CKchapterC}[3]{
  \chapter{#1}
  \vspace*{-144pt}
  \begin{tikzpicture}[remember picture,overlay]
    \node at (current page.north west) [anchor=west,xshift=16pt, yshift=-48pt] (dummy) {};
    \draw [ color=ckgreen, line width=2pt] (dummy.west) --++ ( 580pt , 0pt);   
    \node at (dummy.west) [anchor=north west, xshift=8pt , yshift=0pt] (chap) { \scshape\fontsize{20}{24}\selectfont #3 };
    \node at (chap.north east) [anchor=north west, xshift=2pt , yshift=0pt,rectangle,top color=ckgreen,bottom color=ckgreen!80!white ] {\sffamily\Huge\bfseries\scshape\fontsize{50}{60}\selectfont\color{black} {\thechapter}  };
  \end{tikzpicture}
  \vspace*{16pt}  
  \begin{flushright}  { #2 \normalfont\sffamily\bfseries\color{black} ~ ~ ~ ~ ~ ~ ~ ~ #1 } \end{flushright}
  \vspace*{0pt}
}

% Sometimes content isn't a chapter, it's a concept.
\newcommand{\CKconcept}[2]{
  \chapter{#1}
  \vspace*{-144pt}
  \begin{tikzpicture}[remember picture,overlay]
  \node at (current page.north west) [anchor=west,xshift=0.75in, yshift=-48pt] (dummy) {};
    \draw [ color=ckgreen, line width=2pt] (dummy.west) --++ ( 7.0in , 0pt);   
    \node at (dummy.west) [anchor=north west, xshift=8pt , yshift=0pt] (chap) { \scshape\fontsize{20}{24}\selectfont {Concept} };
    \node at (chap.north east) [anchor=north west, xshift=2pt , yshift=0pt,rectangle,top color=ckgreen,bottom color=ckgreen!80!white ] {\sffamily\Huge\bfseries\scshape\fontsize{50}{60}\selectfont\color{black} {\thechapter}  };
  \end{tikzpicture}
  \vspace*{16pt}  
  \begin{flushright}  { #2 \normalfont\sffamily\bfseries\color{black} ~ ~ ~ ~ ~ ~ ~ ~ ~ \NOhyph{#1}} \end{flushright}
  \vspace*{0pt}
}




%% Title format for section... this only used to wrap *Section functions.
%% Will increment for the TOC and footer... actual display handeled in *Section.
\titleformat{\section}
  { \color{white}\tiny } {} {} {} {}

%% Long Fancy Section: For Long titles that may spill over, use this function
%% \Custom section
%%%\newcommand{\CSection}[2]{
%%%  \section{#1}
%%%  % and now format the header according to spec.
%%%  \begin{tikzpicture}[remember picture,overlay]
%%%    %\fill[left color=ckgreen,right color=white, top color = white,bottom color=ckgreen] (-60pt, 32pt) rectangle (1.5\paperwidth ,0.25cm);
%%%    \shade[left color=ckgreen,right color=white ] (-60pt, 32pt) rectangle (\paperwidth ,0.25cm);
%%%    \node[anchor=west,xshift=0,yshift=0 \paperwidth,rectangle,
%%%      rounded corners=20pt,inner sep=11pt,fill=ckorange]
%%%         {#2\normalfont\sffamily\bfseries\scshape\color{white}{\Huge\thesection} \ #1};
%%%  \end{tikzpicture}
%%%  \vspace*{.2in}
%%%}

\titleformat{\section}
  { \color{white}\tiny } {} {} {} {}

%% Long Fancy Section: For Long titles that may spill over, use this function                                                                                                                                                                
%% \Custom section                                                                                                                                                                                                                           
\newcommand{\CSection}[2]{
  \section{#1}
  \begin{minipage}[t]{\textwidth}
  \begin{tikzpicture}[remember picture,overlay]
    \node at [ anchor= west, xshift=0.0in, yshift=20pt ] (dummy) {};
    %\node at (current page.north west) [anchor=west,xshift=.35in, yshift=-48pt] (dummy) {};
    \draw[ color=ckorange, line width=2pt] (dummy.west) --++ ( 7.0in , 0pt);
    \node at (dummy.west) [anchor=north west, xshift=8pt , yshift=0pt] (chap) {}; %{ \scshape\fontsize{14}{16}\selectfont Section };
    \node at (chap.north east) [anchor=north west, xshift= 25pt,yshift=0pt,rectangle,top color=ckorange,bottom color=ckorange!80!white ] (thenum)
        {\sffamily\Huge\bfseries\selectfont\color{black} {\thesection}  };
    \node at (thenum.north east) [anchor=north west, xshift=0pt, yshift=-2pt, text width=6in] (sectitle) {#2\normalfont\sffamily\bfseries\color{black}\NOhyph{#1}};
  \end{tikzpicture}
  \vspace*{-22pt}
  %\begin{flushleft} {  #2\normalfont\sffamily\bfseries\color{black} ~ ~ ~ ~ ~ ~ #1 } \end{flushleft}
  \vspace*{0pt}
  \end{minipage}
}



%\titleformat{ command }[ shape ]{ format }{ label }{ sep }{ before }[ after ]
%% Set up the section and subsection fonts/colors
%\titleformat{\subsection}[block]{}
%{}{}%{\thesubsection}{10pt}
%{ \noindent
%\color{ckgreen}{\titlerule[2pt]} \\
%\noindent\bfseries\Large\sffamily\color{ckorange}{ #1}
%}

%% Subsection is the h2 level
\titleformat{\subsection}[block]
{\Large\sffamily}
{\color{ckgreen}\titlerule[2pt] \\
\bfseries\color{ckorange}\Large\sffamily{#1}}
{10pt}
{}

%% Subsubsection is the h3 level
\titleformat{\subsubsection}[block]
{} %\large\sffamily\color{ckorange}}
{}%{\bfseries\thesubsubsection #1  }}
{}
{\bfseries\large\sffamily\color{ckorange} #1 }

%% Paragraph is the h4 level: overload function
% Alternative paragraph command
\makeatletter
\renewcommand\paragraph{\@startsection{paragraph}{4}{\z@}%
  {-3.25ex\@plus -1ex \@minus -.2ex}%
  {1.5ex \@plus .2ex}%
  {\normalfont\large\sffamily\bfseries}}
\makeatother

%% Paragraph is the h5 level: overload function
% Alternative subparagraph command
\makeatletter
\renewcommand\subparagraph{\@startsection{subparagraph}{4}{\z@}%
  {-3.25ex\@plus -1ex \@minus -.2ex}%
  {1.5ex \@plus .2ex}%
  {\normalfont\normalsize\sffamily\bfseries}}
\makeatother



%%%% Figure short cuts here
%% need this is make figure show up in multicol
\floatstyle{plain}
\newfloat{ckfloat}{!hbpt}{lop}[chapter]
\floatname{ckfloat}{CKfloat}


\makeatletter
\newenvironment{tablehere}
  {\def\@captype{table}}
  {}

\newenvironment{longtablehere}
  {\def\@captype{longtable}}
  {}

\newenvironment{figurehere}
  {\def\@captype{figure}}
  {}

\newenvironment{imagehere}
  {\def\@captype{figure*}}
  {}
\makeatother

%% Wrapper macro for figure stuff.  Make sure thw following are set in template.tex file
%\newdimen\ckfigwidth
%\ckfigwidth=3in
%\newdimen\ckfigwidthW
%\ckfigwidthW=6in

%\newcommand{\ckfigWpre}{
%\end{multicols}
%\clearpage
%}

%\newcommand{\ckfigWpost}{
%\begin{mulitcols}{2}
%}



\newcommand{\ckfig}[4]{ 
%\vspace{.25in}

\raisebox{1ex-\height}{
\begin{minipage}{\columnwidth}
\\ 
\begin{figurehere}
	\begin{center}
	\includegraphics[width=.9\columnwidth]{#1}
 	\caption[#2]{#3}
	\label{#4}
        \end{center}
\end{figurehere}
\\
\end{minipage}
}\vspace*{6}
%\vspace{.25in}
}

\newcommand{\ckfigw}[4]{ 
\ckfigWpre
%\vspace{.25in}
\begin{figure}
	\begin{center}
	\includegraphics[width=\ckfigwidthW]{#1}
	\caption[#2]{#3}
	\label{#4}
        \end{center}
\end{figure}
%\vspace{.25in}
\ckfigWpost
}


\captionsetup[figure]{labelformat=empty, labelsep=none}


\newcommand{\ckfigb}[5]{
\noindent
%\raisebox{1ex-\height}{
%\begin{center}                                  % No indent to keep the entire figure aligned
\begin{minipage}{\textwidth}             % Minipage will ensure caption and image are on same page/column
    \begin{minipage}{\columnwidth}                                       
    \captionof{figure}{                    % Add to caption/label the proper way here
         \label{#4}}
    \begin{center}                         % Center the image
        \includegraphics[width=#5]{#1}  % Include the actual image file here
    \end{center}
    \end{minipage}
    \begin{minipage}{\columnwidth}
    \begin{center}
    \vspace*{10pt}                          % Spacing between figure caption and figure
    \ckCaptionb{#2}{#3}                    % Put the custom figure caption in here
    \end{center}
    \vspace*{1em}
    \end{minipage}
\end{minipage} 
%\end{center}%}
%\vspace{-32pt}  % This causes things to break on rare occasions
}

\newlength{\figcapL}
\newcommand{\ckfigwbc}[7]{
\setlength{\figcapL}{#6}
\ckfigWpre
\newline
\noindent
%\begin{ckfloat}
\begin{minipage}{\textwidth}             % Minipage will ensure caption and image are on same page/column
    %\vspace*{20pt}                                          
    \begin{minipage}{#5}
    \begin{minipage}{#5}                % No bigger than 4.5in
    \captionof{figure}{                    % Add to caption/label the proper way here... place at top of image
         \label{#4}}
    \begin{center}                         % Center the image
        \if\relax#7\relax     % #7 is reserved for url links: if exists, then execute the else
            \includegraphics[width=#5]{#1}  % Include the actual image file here
        \else
            \href{#7}{\includegraphics[width=#5]{#1}}
        \fi
    \end{center}
    \end{minipage}
    \end{minipage}
    \hspace*{20pt}
    \begin{minipage}{#6}
    \centering        
    \begin{minipage}{.8\figcapL}%{2.3in}
        \ckCaptionb{#2}{#3}                    % Put the custom figure caption in here (already incremented)
    \end{minipage}
    \end{minipage}
\end{minipage}%}
%\end{ckfloat}
\ckfigWpost
}



\newcommand{\ckfigwb}[6]{
\ckfigWpre
\newline
\noindent


%\raisebox{1ex-\height}{    %  No indent to keep the entire figure aligned
%\begin{ckfloat}
\begin{minipage}{\textwidth}
\begin{center}
\begin{minipage}{#5}             % Minipage will ensure caption and image are on same page/column
    \captionof{figure}{                    % Add to caption/label the proper way here
         \label{#4}}
    \begin{center}                         
        % #6 is reserved for url links, if one exists, then execute the \else part:
        \if\relax#6\relax 
          \includegraphics[width=#5]{#1}  % Include the actual image file here
        \else 
          \href{#6}{\includegraphics[width=#5]{#1}}
        \fi
    \end{center}
\vspace*{5pt}             % Spacing between figure caption and figure
\ckCaptionb{#2}{#3}       % Put the custom figure caption in here
\end{minipage}
\end{center}%}
\end{minipage}
\ckfigWpost
%\end{ckfloat}
}


\newcommand{\iFrame}[7]{
\vspace*{10pt}
\noindent
\begin{center}
        \begin{minipage}{#7}
                \begin{center}
                \href{#1}
                { \mbox{
                  \includegraphics[width=#7]{#2}
                }}
                \end{center}
                \vspace*{5pt}
                \ckCaptionC{#3}{#4}{#6}
        \end{minipage}
\end{center}
}

\newcommand{\iFrameW}[9]{
\vspace*{10pt}
\noindent
\begin{center}
\begin{minipage}{7in}
        \begin{center}
        \begin{minipage}{#7}
                \href{#1}
                { \mbox{
                  \includegraphics[width=#7]{#2}
                }}
        \end{minipage}
        \hspace*{20pt}
        \begin{minipage}{3.5in}
                \vspace*{0.2in}
                \centering
                \ckCaptionIframe{#3}{#4}{#6}{#9}
        \end{minipage}
        \end{center}
\end{minipage}
    \label{#8}
\end{center}
}

\newcommand{\iFrameT}[7]{
\vspace*{10pt}
\noindent
\begin{center}
\begin{minipage}{3.2in}
        \begin{center}
        \begin{minipage}{#7}
                \href{#1}
                { \mbox{
                  \includegraphics[width=#7]{#2}
                }}
        \end{minipage}
        \hspace*{0.1in}
        \begin{minipage}{#7}
                \vspace*{0.2in}
                \centering
                \ckCaptionC{#3}{#4}{#6}
        \end{minipage}
        \end{center}
\end{minipage}
\end{center}
}




\newcommand{\ckCaptionb}[2]{
\begin{tikzpicture}
    \node (B) [thick, fill=ckgreen!50 ,                                                      % Thickness of outer line, fill color, line color
    rounded corners=0pt]                                                                     % Width of the box and rounded corners                    
    { {\large\sffamily\MakeUppercase{ Figure~\thefigure }}   };                              % Title/header                                                       
    \draw[ultra thick,anchor=west,color=ckorange] (B.north west) -- +(\columnwidth ,0);   % Add horizontal line above
    \draw[ultra thick,anchor=west,color=ckorange] (B.south east) -- (B.north east);          % Add vertical line after label
    %\node (TT) [anchor=south west] at (B.south east) { #1 };                                % Optional short caption
    \node (T1) [anchor=north west] at (B.south west) {                                       % Full caption text here
          \noindent
          \begin{minipage}{.95\columnwidth}
          {\sffamily\small #2}
          \end{minipage} } ;
    \draw[ultra thick,color=ckorange,anchor=north west] (T1.south west) -- +(\columnwidth ,0);  % Add horizontal line below
\end{tikzpicture}  
}

\newcommand{\ckCaptionC}[3]{
\begin{tikzpicture}
    \node (B) [thick, fill=ckgreen!50 ,                                                      % Thickness of outer line, fill color, line color
    rounded corners=0pt]                                                                     % Width of the box and rounded corners                    
    { {\large\sffamily\MakeUppercase{ #3 }}   };                              % Title/header                                                       
    \draw[ultra thick,anchor=west,color=ckorange] (B.north west) -- +(\columnwidth ,0);   % Add horizontal line above
    \draw[ultra thick,anchor=west,color=ckorange] (B.south east) -- (B.north east);          % Add vertical line after label
    %\node (TT) [anchor=south west] at (B.south east) { #1 };                                % Optional short caption
    \node (T1) [anchor=north west] at (B.south west) {                                       % Full caption text here
          \noindent
          \begin{minipage}{.95\columnwidth}
          {\sffamily\small #2} 
          \end{minipage} } ;
    \draw[ultra thick,color=ckorange,anchor=north west] (T1.south west) -- +(\columnwidth ,0);  % Add horizontal line below
\end{tikzpicture}  
}

%% RESEMBLES ckCaptionC, but takes 4 arguments 
\newcommand{\ckCaptionIframe}[4]{
\begin{tikzpicture}
    \node (B) [thick, fill=ckgreen!50 ,                                                      % Thickness of outer line, fill color, line color
    rounded corners=0pt]                                                                     % Width of the box and rounded corners                    
    { {\large\sffamily\MakeUppercase{ #3 }}   };                              % Title/header                                                       
    \draw[ultra thick,anchor=west,color=ckorange] (B.north west) -- +(\columnwidth ,0);   % Add horizontal line above
    \draw[ultra thick,anchor=west,color=ckorange] (B.south east) -- (B.north east);          % Add vertical line after label
    %\node (TT) [anchor=south west] at (B.south east) { #1 };                                % Optional short caption
    \node (T1) [anchor=north west] at (B.south west) {                                       % Full caption text here
          \noindent
          \begin{minipage}{.95\columnwidth}
          {\sffamily\small #2} \\
          {\sffamily\small #4}  % sub caption
          \end{minipage} } ;
    \draw[ultra thick,color=ckorange,anchor=north west] (T1.south west) -- +(\columnwidth ,0);  % Add horizontal line below
\end{tikzpicture}  
}


%% Wrapper macro for figure(no caption) stuff
\newcommand{\ckim}[3]{ 
%\vspace{.25in}
\\
\begin{imagehere}
  \begin{center}
    \includegraphics[width=#2]{#1}
    \label{#3}
  \end{center}
\end{imagehere}
%\vspace{.25in}
}

\newcommand{\ckimC}[5]{ 
%\vspace{.25in}
%\\
\begin{imagehere}
  \begin{center}
    \if\relax#5\relax
      \includegraphics[width=#2]{#1}
    \else
      \href{#5}{\includegraphics[width=#2]{#1}}
    \fi
    \label{#3}
  \end{center}
\end{imagehere}
%\vspace{.25in}
}

\newcommand{\ckigC}[2]{
        \begin{center}
        \includegraphics[width=#2]{#1}
        \end{center}
}



\newcommand{\ckimw}[3]{ 
\ckfigWpre
%\vspace{.25in}
\begin{figure*}
	\centering
	\includegraphics[width=#2]{#1}
	\label{#3}
\end{figure*}
%\vspace{.25in}
\ckfigWpost
}



\newcommand{\cktab}[6]{ 
\begin{minipage}{\columnwidth}
\vspace*{20pt}
\begin{center}
  \begin{tablehere}
    \caption[#1]{#2}     % Caption arguments
    \begin{tabular}{#3}  % Spacing string
      #4                 % Table header
      #5                 % Table contents
      \label{#6}         % Label here
    \end{tabular}
  \end{tablehere}
\end{center}
\end{minipage}
%\vspace{.25in}
}

% 1-fname. 2-spacing string, 3,4-caption, 5-head, 6-body, 7-label, 8-width
\newcommand{\cktabular}[8]{
\begin{filecontents}{#1}
\begin{longtable}{#2}
        \caption[#3]{#4 \label{#7}}
        #5
        #6
\end{longtable}
\end{filecontents}
\LTXtable{#8}{#1}
}

\newcommand{\cktabw}[6]{ 
\cktabWpre
\begin{longtable}
	\centering
        \caption[#1]{#2}
        \begin{tabular}{#3}
          \hline
          #4          % Table header
          \hline
          #5          % Table contents
          \hline
          \label{#6}  % Label here
        \end{tabular}
\end{longtable}
\cktabWpost
}


\newcommand{\cktabww}[6]{ 
%\vspace*{.25in}
%\cktabWWpre
%\begin{landscape}
\begin{table}
	\centering
        \caption[#1]{#2}
        \begin{tabular}{#3}
          \hline
          #4          % Table header
          \hline
          #5          % Table contents
          \hline
          \label{#6}  % Label here
        \end{tabular}
\end{table}
%\end{landscape}
%\cktabWWpost
}

\newcommand{\cktabB}[6]{ 
%\vspace{.25in}
\begin{center}
  \begin{tablehere}
    \caption[#1]{#2 \label{#6}}     % Caption arguments and label
    \begin{tabular}{#3}  % Spacing string
      \hline
      #4                 % Table header
      \hline
      \hline
      #5                 % Table contents
      \hline
    \end{tabular}
  \end{tablehere}
\end{center}
%\vspace{.25in}
}

\newcommand{\cktabwB}[6]{ 
\cktabWpre
\begin{table}
	\centering
        \caption[#1]{#2 \label{#6}}
        \begin{tabular}{#3}
          \hline
          #4          % Table header
          \hline
          \hline
          #5          % Table contents
          \hline
        \end{tabular}
\end{table}
\cktabWpost
}


\newcommand{\cktabwwB}[6]{ 
\vspace{.25in}
\cktabWWpre
\begin{landscape}
\begin{table}
	\centering
        \caption[#1]{#2 \label{#6}}
        \begin{tabular}{#3}
          \hline
          #4          % Table header
          \hline
          \hline
          #5          % Table contents
          \hline
        \end{tabular}
\end{table}
\end{landscape}
\cktabWWpost
}



%% Element Box Macro... use Tikz and take 3 arguments
%\newdimen\Eboxwidth
%\ckfigwidth=3in

\newcommand{\Ebox}[3]{ 
\begin{tikzpicture}
    \node[ultra thick, fill=#1!20 , draw=#1,                % Thickness of outer line, fill color, line color
    rectangle split, rectangle split parts=2,               % rectangle split node in 2 parts
    drop shadow={shadow xshift=.1in, shadow yshift=-.1in},  % Overall width.. This needs to be 
    text width=\Eboxwidth,rounded corners=10pt]                    % Width of the box and rounded corners
        {
        \vspace*{.05in}
        {\large\sffamily\bfseries{#2}}               % Title/header
        \vspace*{.05in}
        \nodepart{second}
        \vspace*{.05in}
        #3                                                  % Body of element
        \vspace*{.05in}
        }; 
\end{tikzpicture} 
} 

% Enhanced Ebox macro. This is an environment... not a command.
\newcommand{\tmpColor}{}
\newsavebox{\EboxRoundedBox}
\newsavebox{\EboxRoundedBoxHead}
\newenvironment{EEbox}[2]%
   {\renewcommand{\tmpColor}{#2}%
    \begin{lrbox}{\EboxRoundedBoxHead}
        \begin{minipage}{\textwidth}
        {\large\sffamily\bfseries{#1}}
        \end{minipage}
    \end{lrbox}
    \begin{lrbox}{\EboxRoundedBox}
       \begin{minipage}{\textwidth}}%
   {   \end{minipage}
    \end{lrbox}
    \begin{center}
    \begin{tikzpicture}%
       \draw node[ultra thick,draw=\tmpColor,fill=\tmpColor!20,rounded corners=10pt,%
             rectangle split, rectangle split parts=2,               % rectangle split node in 2 parts                                                                                                                                                
                        drop shadow={shadow xshift=.1in, shadow yshift=-.1in},
             inner sep=2ex,text width=\textwidth]%
             {\usebox{\EboxRoundedBoxHead}
             \nodepart{second}
             \usebox{\EboxRoundedBox}};
    \end{tikzpicture}
    \end{center}}



%% URL macro to protect line+page break combo which will break url
\newcommand{\ckurl}[3]{
\\
\begin{minipage}{#3}
        \vspace{3pt}
        %\url{#1}
        \href{#1}{#2}
        \vspace*{3pt}
\end{minipage}
}

%% Highlight macro here
\newlength{\HLlen}
\newcommand{\highlight}[2]{
        \settowidth{\HLlen}{#2}
        \pgfmathsetlength{\HLlen}{min(\HLlen,\textwidth)}
        \noindent\colorbox{#1}{\noindent\parbox{\HLlen}{#2}}
} 


\newcommand{\ckHline}{
\noindent
\begin{tikzpicture}
        \draw[color=black,line width=2pt] (0,0) --++ (\textwidth,0);
\end{tikzpicture}
}

\newcommand\ckhl[2]{%
    \tikz[baseline,%
      decoration={random steps,amplitude=1pt,segment length=15pt},%
      outer sep=-15pt, inner sep = 0pt%
    ]%
   \node[decorate,rectangle,fill=#1,anchor=text]{#2\xspace};%
}%


% No hypenation in this command
\newcommand{\NOhyph}[1]{
\lefthyphenmin=60\righthyphenmin=60{#1}\lefthyphenmin=3\righthyphenmin=3}



%% Bold table text macro
\newcommand{\bftab}[1]{\bf$\mathbf{#1}$}
\newcommand{\bftabt}[1]{\bf #1}


%% Macros for the book_head (authors,editors,contributors,etc.)

\newcommand{\nameBox}[2]{
\begin{centering}
    {\color{ckgreen}\titlerule[2pt]} 
    \\
    \vspace*{18pt}
    { \huge\color{black}
    #1
    }
    \\
    \vspace*{18pt}
% 
        {\color{ckgreen}\titlerule[2pt]} 
\end{centering}

{\large\sffamily\scshape\bfseries\color{ckorange} #2 } 
}

\newcommand{\ckMP}[2]{
\begin{minipage}{#1}
        #2
\end{minipage}
}


\newcommand{\ckCopyPage}[2]{
\begin{tikzpicture}
        \node (A) []{
              \ckMP{4.0in}{#1}
              };
        \draw[color=black] (A.north east)+(.025\textwidth , 0) --+ (.025\textwidth ,-\textheight);
        \node [anchor=north west,xshift=.05\textwidth] at (A.north east) {
              \ckMP{2.0in}{#2}
              };
\end{tikzpicture}
}


\usepackage[normalem]{ulem}


\usepackage[sectionbib]{chapterbib}
\usepackage{natbib}
\renewcommand{\bibname}{Image Sources}
\makeatletter
\renewcommand\@biblabel[1]{(#1)}
\makeatother


\renewcommand{\LARGE}{\fontsize{16}{18}\selectfont}
%\titleformat{\chapter}[display]
%{\raggedright \normalfont\Huge\bfseries}{\chaptertitlename\ \thechapter}{24pt}{\Huge}


\newcommand{\Beta}{B}
\newcommand{\unit}[1]{\ensuremath{\, \mathrm{#1}}}


\DeclareMathSizes{11}{12}{9}{6}
%% Bug 2950, try to reduce vertical margins by half
\newcounter{tempc} \newcounter{tempcc}
\setlength\textheight %
%%	{10in-\topskip}
        {9.5in-\topskip}
\setcounter{tempc}{\textheight}
\setcounter{tempcc}{\baselineskip}
\setcounter{tempc} %
	{\value{tempc}/\value{tempcc}}
\setlength\textheight{\baselineskip*\value{tempc}+\topskip}


%% Margins and spacings
% Left - Right stuff
\setlength{\hoffset}{-1.0in}                    %% Zero out the left side
\setlength{\oddsidemargin}{.75in}		%% Left start of text
\setlength{\evensidemargin}{.75in}		%% Right start of text
\setlength{\columnsep}{0.5in}			%% Gap between the columns
\setlength{\textwidth}{7.0in}			%% Total text width... = (8.5 - odd - even ) / 2

%% Top Bottom stuff
\setlength{\voffset}{-1.0in} 
\setlength{\topmargin}{.25in}  % .75
\setlength{\headheight}{12pt}
\setlength{\headsep}{.25in}
\setlength{\textheight}{9.65in}
%\setlength{\footskip}{.25in}


%% For custom footers and Headers
\usepackage{fancyhdr}
\fancypagestyle{plain}{%
\fancyhf{} % clear all header and footer fields
\fancyfoot[LE,RO]{\thepage  } 
\fancyhead[LO,RE]{\href{http://www.ck12.org}{ \cklogo } }
\renewcommand{\chaptername}{Chapter}
\fancyhead[LE]{\nouppercase{\slshape  \rightmark}}
\fancyhead[RO]{\nouppercase{\leftmark \slshape}} %RO=right odd, RE=right even
\renewcommand{\headrulewidth}{0pt}
\renewcommand{\footrulewidth}{0pt}
}

\fancypagestyle{concept}{%
\fancyhf{} % clear all header and footer fields
\fancyfoot[LE,RO]{\thepage  } 
\fancyhead[LO,RE]{\href{http://www.ck12.org}{ \cklogo } }
\renewcommand{\chaptername}{Concept}
\fancyhead[LE]{\nouppercase{\slshape  \rightmark}}
\fancyhead[RO]{\nouppercase{\leftmark \slshape}} %RO=right odd, RE=right even
\renewcommand{\headrulewidth}{0pt}
\renewcommand{\footrulewidth}{0pt}
}

\pagestyle{plain}


%\usepackage{underscore}  % this package breaks file path with underscores

%%%% Define some colors
\definecolor{ckorange}{RGB}{207, 87, 38}      % ck12 orange
\definecolor{lckorange}{RGB}{231, 156, 126}   % ck12 light orange

\definecolor{ckgreen}{RGB}{167, 203, 101}     % ck12 green
\definecolor{lckgreen}{RGB}{201, 223, 159}    % ck12 light green



% From Rosetta Stone: Missing colors
\definecolor{aqua}   {HTML}{00FFFF}
\definecolor{lime}   {HTML}{00FF00} 
\definecolor{green}  {HTML}{008000} 
\definecolor{navy}   {HTML}{000080}  
\definecolor{olive}  {HTML}{808000} 
\definecolor{silver} {HTML}{C0C0C0}
\definecolor{teal}   {HTML}{008080}
\definecolor{ckgrey} {HTML}{808080}
\definecolor{grey}   {HTML}{808080}
\definecolor{gray}   {HTML}{808080}
\definecolor{fuchsia}{HTML}{FF00FF}
\definecolor{purple} {HTML}{800080}
\definecolor{maroon} {HTML}{800000}
\definecolor{orange} {HTML}{FFA500}
\definecolor{darkorange} {HTML}{FF8C00}


% Multi-cols parameters:
\setlength\premulticols{10\baselineskip}  % Ensure there is enough space at the bottom  (Fixes most image overflows...)
\setcounter{columnbadness}{7000}
\setcounter{finalcolumnbadness}{7000}


%\tolerance=1
%\emergencystretch=\maxdimen
%\hyphenpenalty=10000
%\hbadness=10000

\usepackage{hyphenat}

% Alter some LaTeX defaults for better treatment of figures:
% See p.105 of ``TeX Unbound'' for suggested values.
%   General parameters, for ALL pages:
%\renewcommand{\topfraction}{0.9}    % max fraction of floats at top
%\renewcommand{\bottomfraction}{0.8} % max fraction of floats at bottom
%   Parameters for TEXT pages (not float pages):
%\setcounter{topnumber}{2}
%\setcounter{bottomnumber}{2}
%\setcounter{totalnumber}{4}     % 2 may work better
%\setcounter{dbltopnumber}{2}    % for 2-column pages
%\renewcommand{\dbltopfraction}{0.9}   % fit big float above 2-col. text
%\renewcommand{\textfraction}{0.07}    % allow minimal text w. figs
%%   Parameters for FLOAT pages (not text pages):
%\renewcommand{\floatpagefraction}{0.7}   % require fuller float pages
% N.B.: floatpagefraction MUST be less than topfraction !!
%\renewcommand{\dblfloatpagefraction}{0.7}   % require fuller float pages
